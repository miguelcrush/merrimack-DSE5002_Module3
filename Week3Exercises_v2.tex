% Options for packages loaded elsewhere
\PassOptionsToPackage{unicode}{hyperref}
\PassOptionsToPackage{hyphens}{url}
%
\documentclass[
]{article}
\usepackage{amsmath,amssymb}
\usepackage{iftex}
\ifPDFTeX
  \usepackage[T1]{fontenc}
  \usepackage[utf8]{inputenc}
  \usepackage{textcomp} % provide euro and other symbols
\else % if luatex or xetex
  \usepackage{unicode-math} % this also loads fontspec
  \defaultfontfeatures{Scale=MatchLowercase}
  \defaultfontfeatures[\rmfamily]{Ligatures=TeX,Scale=1}
\fi
\usepackage{lmodern}
\ifPDFTeX\else
  % xetex/luatex font selection
\fi
% Use upquote if available, for straight quotes in verbatim environments
\IfFileExists{upquote.sty}{\usepackage{upquote}}{}
\IfFileExists{microtype.sty}{% use microtype if available
  \usepackage[]{microtype}
  \UseMicrotypeSet[protrusion]{basicmath} % disable protrusion for tt fonts
}{}
\makeatletter
\@ifundefined{KOMAClassName}{% if non-KOMA class
  \IfFileExists{parskip.sty}{%
    \usepackage{parskip}
  }{% else
    \setlength{\parindent}{0pt}
    \setlength{\parskip}{6pt plus 2pt minus 1pt}}
}{% if KOMA class
  \KOMAoptions{parskip=half}}
\makeatother
\usepackage{xcolor}
\usepackage[margin=1in]{geometry}
\usepackage{color}
\usepackage{fancyvrb}
\newcommand{\VerbBar}{|}
\newcommand{\VERB}{\Verb[commandchars=\\\{\}]}
\DefineVerbatimEnvironment{Highlighting}{Verbatim}{commandchars=\\\{\}}
% Add ',fontsize=\small' for more characters per line
\usepackage{framed}
\definecolor{shadecolor}{RGB}{248,248,248}
\newenvironment{Shaded}{\begin{snugshade}}{\end{snugshade}}
\newcommand{\AlertTok}[1]{\textcolor[rgb]{0.94,0.16,0.16}{#1}}
\newcommand{\AnnotationTok}[1]{\textcolor[rgb]{0.56,0.35,0.01}{\textbf{\textit{#1}}}}
\newcommand{\AttributeTok}[1]{\textcolor[rgb]{0.13,0.29,0.53}{#1}}
\newcommand{\BaseNTok}[1]{\textcolor[rgb]{0.00,0.00,0.81}{#1}}
\newcommand{\BuiltInTok}[1]{#1}
\newcommand{\CharTok}[1]{\textcolor[rgb]{0.31,0.60,0.02}{#1}}
\newcommand{\CommentTok}[1]{\textcolor[rgb]{0.56,0.35,0.01}{\textit{#1}}}
\newcommand{\CommentVarTok}[1]{\textcolor[rgb]{0.56,0.35,0.01}{\textbf{\textit{#1}}}}
\newcommand{\ConstantTok}[1]{\textcolor[rgb]{0.56,0.35,0.01}{#1}}
\newcommand{\ControlFlowTok}[1]{\textcolor[rgb]{0.13,0.29,0.53}{\textbf{#1}}}
\newcommand{\DataTypeTok}[1]{\textcolor[rgb]{0.13,0.29,0.53}{#1}}
\newcommand{\DecValTok}[1]{\textcolor[rgb]{0.00,0.00,0.81}{#1}}
\newcommand{\DocumentationTok}[1]{\textcolor[rgb]{0.56,0.35,0.01}{\textbf{\textit{#1}}}}
\newcommand{\ErrorTok}[1]{\textcolor[rgb]{0.64,0.00,0.00}{\textbf{#1}}}
\newcommand{\ExtensionTok}[1]{#1}
\newcommand{\FloatTok}[1]{\textcolor[rgb]{0.00,0.00,0.81}{#1}}
\newcommand{\FunctionTok}[1]{\textcolor[rgb]{0.13,0.29,0.53}{\textbf{#1}}}
\newcommand{\ImportTok}[1]{#1}
\newcommand{\InformationTok}[1]{\textcolor[rgb]{0.56,0.35,0.01}{\textbf{\textit{#1}}}}
\newcommand{\KeywordTok}[1]{\textcolor[rgb]{0.13,0.29,0.53}{\textbf{#1}}}
\newcommand{\NormalTok}[1]{#1}
\newcommand{\OperatorTok}[1]{\textcolor[rgb]{0.81,0.36,0.00}{\textbf{#1}}}
\newcommand{\OtherTok}[1]{\textcolor[rgb]{0.56,0.35,0.01}{#1}}
\newcommand{\PreprocessorTok}[1]{\textcolor[rgb]{0.56,0.35,0.01}{\textit{#1}}}
\newcommand{\RegionMarkerTok}[1]{#1}
\newcommand{\SpecialCharTok}[1]{\textcolor[rgb]{0.81,0.36,0.00}{\textbf{#1}}}
\newcommand{\SpecialStringTok}[1]{\textcolor[rgb]{0.31,0.60,0.02}{#1}}
\newcommand{\StringTok}[1]{\textcolor[rgb]{0.31,0.60,0.02}{#1}}
\newcommand{\VariableTok}[1]{\textcolor[rgb]{0.00,0.00,0.00}{#1}}
\newcommand{\VerbatimStringTok}[1]{\textcolor[rgb]{0.31,0.60,0.02}{#1}}
\newcommand{\WarningTok}[1]{\textcolor[rgb]{0.56,0.35,0.01}{\textbf{\textit{#1}}}}
\usepackage{graphicx}
\makeatletter
\def\maxwidth{\ifdim\Gin@nat@width>\linewidth\linewidth\else\Gin@nat@width\fi}
\def\maxheight{\ifdim\Gin@nat@height>\textheight\textheight\else\Gin@nat@height\fi}
\makeatother
% Scale images if necessary, so that they will not overflow the page
% margins by default, and it is still possible to overwrite the defaults
% using explicit options in \includegraphics[width, height, ...]{}
\setkeys{Gin}{width=\maxwidth,height=\maxheight,keepaspectratio}
% Set default figure placement to htbp
\makeatletter
\def\fps@figure{htbp}
\makeatother
\setlength{\emergencystretch}{3em} % prevent overfull lines
\providecommand{\tightlist}{%
  \setlength{\itemsep}{0pt}\setlength{\parskip}{0pt}}
\setcounter{secnumdepth}{-\maxdimen} % remove section numbering
\ifLuaTeX
  \usepackage{selnolig}  % disable illegal ligatures
\fi
\usepackage{bookmark}
\IfFileExists{xurl.sty}{\usepackage{xurl}}{} % add URL line breaks if available
\urlstyle{same}
\hypersetup{
  pdftitle={Week 3 Exercises},
  pdfauthor={HD Sheets},
  hidelinks,
  pdfcreator={LaTeX via pandoc}}

\title{Week 3 Exercises}
\author{HD Sheets}
\date{10/1/2024}

\begin{document}
\maketitle

Please complete all exercises below. You may use any library that we
have covered in class UP TO THIS POINT.

\begin{Shaded}
\begin{Highlighting}[]
\FunctionTok{library}\NormalTok{(stringr)}
\end{Highlighting}
\end{Shaded}

For each function, show that it works, by using the provided data as a
test and by feeding in some test data that you create to test your
function

Add comments to your function to explain what each line is doing

1.) Write a function that takes in a string with a person's name in the
form

``Sheets, Dave''

and returns a string of the form

``Dave Sheets''

Note:

-assume no middle initial ever -remove the comma -be sure there is white
space between the first and last name

You will probably want to use stringr

\begin{Shaded}
\begin{Highlighting}[]
\NormalTok{name\_in}\OtherTok{=}\StringTok{"Sheets, Dave"}

\NormalTok{reorder\_name}\OtherTok{\textless{}{-}}\ControlFlowTok{function}\NormalTok{(last\_first)\{}
  \DocumentationTok{\#\# split the name string by the comma}
\NormalTok{  parts }\OtherTok{\textless{}{-}} \FunctionTok{str\_split}\NormalTok{(name\_in, }\StringTok{","}\NormalTok{)}
  \DocumentationTok{\#\# sapply (tapply except for vectors) str\_trim against the split string and reverse}
\NormalTok{  res }\OtherTok{\textless{}{-}} \FunctionTok{sapply}\NormalTok{(parts, str\_trim)}
  \DocumentationTok{\#\# concatenate the results}
  \FunctionTok{return}\NormalTok{(}\FunctionTok{cat}\NormalTok{(res[}\DecValTok{2}\NormalTok{], }\StringTok{\textquotesingle{} \textquotesingle{}}\NormalTok{, res[}\DecValTok{1}\NormalTok{], }\AttributeTok{sep=}\StringTok{\textquotesingle{}\textquotesingle{}}\NormalTok{))}
\NormalTok{\}}

\FunctionTok{reorder\_name}\NormalTok{(name\_in)}
\end{Highlighting}
\end{Shaded}

\begin{verbatim}
## Dave Sheets
\end{verbatim}

2.) Write a function that takes in a string of values x, and returns a
data frame with three columns, x, x\^{}2 and the square root of x

\begin{Shaded}
\begin{Highlighting}[]
\NormalTok{x}\OtherTok{=}\FunctionTok{c}\NormalTok{(}\DecValTok{1}\NormalTok{,}\DecValTok{3}\NormalTok{,}\DecValTok{5}\NormalTok{,}\DecValTok{7}\NormalTok{,}\DecValTok{9}\NormalTok{,}\DecValTok{11}\NormalTok{,}\DecValTok{13}\NormalTok{)}

\NormalTok{powers\_df}\OtherTok{\textless{}{-}}\ControlFlowTok{function}\NormalTok{(x)}
\NormalTok{\{}
  \DocumentationTok{\#\# create a data frame consisting of the requested columns and their calculations}
\NormalTok{  df }\OtherTok{\textless{}{-}} \FunctionTok{data.frame}\NormalTok{(}\AttributeTok{x =}\NormalTok{ x, }\AttributeTok{xsq=}\NormalTok{x}\SpecialCharTok{\^{}}\DecValTok{2}\NormalTok{, }\AttributeTok{xsqr=}\NormalTok{x}\SpecialCharTok{\^{}}\FloatTok{0.5}\NormalTok{)}
  \FunctionTok{return}\NormalTok{(df)}
\NormalTok{\}}

\CommentTok{\#your code here}
\FunctionTok{powers\_df}\NormalTok{(x)}
\end{Highlighting}
\end{Shaded}

\begin{verbatim}
##    x xsq     xsqr
## 1  1   1 1.000000
## 2  3   9 1.732051
## 3  5  25 2.236068
## 4  7  49 2.645751
## 5  9  81 3.000000
## 6 11 121 3.316625
## 7 13 169 3.605551
\end{verbatim}

3.) Write in a function that takes in a value x and returns

\begin{verbatim}
y= 0.3x if x<0
y=0.5x if x>=0

This is a variant on a relu function as used in some neural networks.
\end{verbatim}

\begin{Shaded}
\begin{Highlighting}[]
\NormalTok{func\_3 }\OtherTok{\textless{}{-}} \ControlFlowTok{function}\NormalTok{(x)\{}
  \DocumentationTok{\#\# if x \textless{} 0 then return the specified string}
  \ControlFlowTok{if}\NormalTok{(x}\SpecialCharTok{\textless{}}\DecValTok{0}\NormalTok{)\{}
    \FunctionTok{return}\NormalTok{(}\StringTok{"y= 0.3x"}\NormalTok{)}
\NormalTok{  \}}
  \DocumentationTok{\#\# otherwise, return the other specified string}
  \ControlFlowTok{else}\NormalTok{\{}
    \FunctionTok{return}\NormalTok{(}\StringTok{"y=0.5x"}\NormalTok{)}
\NormalTok{  \}}
\NormalTok{\}}

\FunctionTok{func\_3}\NormalTok{(}\DecValTok{1}\NormalTok{)}
\end{Highlighting}
\end{Shaded}

\begin{verbatim}
## [1] "y=0.5x"
\end{verbatim}

\begin{Shaded}
\begin{Highlighting}[]
\DocumentationTok{\#\# OR}
\NormalTok{func\_3 }\OtherTok{\textless{}{-}} \ControlFlowTok{function}\NormalTok{(x)\{}
  \DocumentationTok{\#\# if x \textless{} 0 then return x * 0.3}
  \ControlFlowTok{if}\NormalTok{(x}\SpecialCharTok{\textless{}}\DecValTok{0}\NormalTok{)\{}
    \FunctionTok{return}\NormalTok{(}\FloatTok{0.3} \SpecialCharTok{*}\NormalTok{ x)}
\NormalTok{  \}}
  \DocumentationTok{\#\# otherwise return x * 0.5}
  \ControlFlowTok{else}\NormalTok{\{}
    \FunctionTok{return}\NormalTok{(}\FloatTok{0.5} \SpecialCharTok{*}\NormalTok{ x)}
\NormalTok{  \}}
\NormalTok{\}}

\FunctionTok{func\_3}\NormalTok{(}\DecValTok{1}\NormalTok{)}
\end{Highlighting}
\end{Shaded}

\begin{verbatim}
## [1] 0.5
\end{verbatim}

4.) Write a function that takes in a value x and returns the first power
of two greater than x (use a While loop)

\begin{Shaded}
\begin{Highlighting}[]
\NormalTok{func\_4 }\OtherTok{\textless{}{-}} \ControlFlowTok{function}\NormalTok{(x)\{}
  \DocumentationTok{\#\# declare a var to track the current power}
\NormalTok{  power}\OtherTok{\textless{}{-}} \DecValTok{0}
  \DocumentationTok{\#\# keep incrementing the power var until 2\^{}power \textgreater{} x}
  \ControlFlowTok{while}\NormalTok{(}\DecValTok{2}\SpecialCharTok{\^{}}\NormalTok{power }\SpecialCharTok{\textless{}}\NormalTok{ x)\{}
\NormalTok{    power }\OtherTok{\textless{}{-}}\NormalTok{ power}\SpecialCharTok{+}\DecValTok{1}
\NormalTok{  \}}
  
  \FunctionTok{return}\NormalTok{(}\DecValTok{2}\SpecialCharTok{\^{}}\NormalTok{power)}
\NormalTok{\}}

\FunctionTok{func\_4}\NormalTok{(}\DecValTok{1050}\NormalTok{)}
\end{Highlighting}
\end{Shaded}

\begin{verbatim}
## [1] 2048
\end{verbatim}

\begin{enumerate}
\def\labelenumi{\arabic{enumi})}
\setcounter{enumi}{4}
\tightlist
\item
  Two Sum - Write a function named two\_sum()
\end{enumerate}

Given a vector of integers nums and an integer target, return indices of
the two numbers such that they add up to target.

You may assume that each input would have exactly one solution, and you
may not use the same element twice.

You can return the answer in any order.

Example 1:

Input: nums = {[}2,7,11,15{]}, target = 9 Output: {[}0,1{]} Explanation:
Because nums{[}0{]} + nums{[}1{]} == 9, we return {[}0, 1{]}. Example 2:

Input: nums = {[}3,2,4{]}, target = 6 Output: {[}1,2{]} Example 3:

Input: nums = {[}3,3{]}, target = 6 Output: {[}0,1{]}

Constraints:

2 \textless= nums.length \textless= 104 --109 \textless= nums{[}i{]}
\textless= 109 --109 \textless= target \textless= 109 Only one valid
answer exists.

\emph{Note: For the first problem I want you to use a brute force
approach (loop inside a loop)}

\emph{The brute force approach is simple. Loop through each element x
and find if there is another value that equals to target -- x}

\emph{Use the function seq\_along to iterate}

\begin{Shaded}
\begin{Highlighting}[]
\NormalTok{two\_sum }\OtherTok{\textless{}{-}} \ControlFlowTok{function}\NormalTok{(nums\_vector,target)\{}
  \DocumentationTok{\#\# iterate over nums\_vector}
  \ControlFlowTok{for}\NormalTok{(i }\ControlFlowTok{in} \FunctionTok{seq\_along}\NormalTok{(nums\_vector))\{}
\NormalTok{    ei }\OtherTok{\textless{}{-}}\NormalTok{ nums\_vector[i]}
    \DocumentationTok{\#\# iterate over nums\_vector again}
    \ControlFlowTok{for}\NormalTok{(j }\ControlFlowTok{in} \FunctionTok{seq\_along}\NormalTok{(nums\_vector))\{}
      \DocumentationTok{\#\# and you may not use the same element twice.}
      \ControlFlowTok{if}\NormalTok{(i}\SpecialCharTok{==}\NormalTok{ j)\{}
        \ControlFlowTok{next} \DocumentationTok{\#\# similar to \textasciigrave{}continue\textasciigrave{} in other langs}
\NormalTok{      \}}
\NormalTok{      ej }\OtherTok{\textless{}{-}}\NormalTok{ nums\_vector[j]}
      \DocumentationTok{\#\# if the numbers at each index add up to target, return the indecies}
      \ControlFlowTok{if}\NormalTok{(ej }\SpecialCharTok{+}\NormalTok{ ei }\SpecialCharTok{==}\NormalTok{ target)\{}
        \FunctionTok{return}\NormalTok{(}\FunctionTok{c}\NormalTok{(i, j))}
\NormalTok{      \}}
\NormalTok{    \}}
\NormalTok{  \}}
 
\NormalTok{\}}


\CommentTok{\# Test code}
\NormalTok{nums\_vector }\OtherTok{\textless{}{-}} \FunctionTok{c}\NormalTok{(}\DecValTok{5}\NormalTok{,}\DecValTok{7}\NormalTok{,}\DecValTok{12}\NormalTok{,}\DecValTok{34}\NormalTok{,}\DecValTok{6}\NormalTok{,}\DecValTok{10}\NormalTok{,}\DecValTok{8}\NormalTok{,}\DecValTok{9}\NormalTok{)}
\NormalTok{target }\OtherTok{\textless{}{-}} \DecValTok{13}
 
\NormalTok{z}\OtherTok{=}\FunctionTok{two\_sum}\NormalTok{(nums\_vector,target)}
\FunctionTok{print}\NormalTok{(z)}
\end{Highlighting}
\end{Shaded}

\begin{verbatim}
## [1] 1 7
\end{verbatim}

\begin{Shaded}
\begin{Highlighting}[]
\CommentTok{\#expected answers}
\CommentTok{\#[1] 1 7}
\CommentTok{\#[1] 2 5}
\CommentTok{\#[1] 5 2}
\end{Highlighting}
\end{Shaded}

6.) Write one piece of code that will use a regex command to extract a
phone number written in the form

\begin{verbatim}
  456-123-2329
  
\end{verbatim}

The sentences to use are located below

use the str\_extract function from stringr

use the same regex search pattern from each

-What does \textbackslash d match to? or alternatively
{[}{[}:digit:{]}{]}

\begin{Shaded}
\begin{Highlighting}[]
\DocumentationTok{\#\# Both are equivalent to the expression [0{-}9]}
\end{Highlighting}
\end{Shaded}

-How do you specify a specific number of repeated characters

\begin{Shaded}
\begin{Highlighting}[]
\DocumentationTok{\#\# x\{n\}, where x is a character n is the number of repeated x}
\end{Highlighting}
\end{Shaded}

\begin{Shaded}
\begin{Highlighting}[]
\NormalTok{a}\OtherTok{=}\StringTok{"Please call me at 456{-}123{-}2329, asap"}
\NormalTok{b}\OtherTok{=}\StringTok{"Hey, we have a code 234 on machine a{-}234{-}12, call me at 678{-}321{-}98766"}
\NormalTok{c}\OtherTok{=}\StringTok{"On 12{-}23{-}2022, Joe over at 122 Turnpike, dialled 912{-}835{-}4756, tell me by 9:02 pm Wed"}

\NormalTok{phone\_regex\_pattern }\OtherTok{\textless{}{-}} \StringTok{"}\SpecialCharTok{\textbackslash{}\textbackslash{}}\StringTok{d\{3\}{-}}\SpecialCharTok{\textbackslash{}\textbackslash{}}\StringTok{d\{3\}{-}}\SpecialCharTok{\textbackslash{}\textbackslash{}}\StringTok{d\{4\}"}
\FunctionTok{str\_extract}\NormalTok{(a, phone\_regex\_pattern)}
\end{Highlighting}
\end{Shaded}

\begin{verbatim}
## [1] "456-123-2329"
\end{verbatim}

\begin{Shaded}
\begin{Highlighting}[]
\FunctionTok{str\_extract}\NormalTok{(b, phone\_regex\_pattern)}
\end{Highlighting}
\end{Shaded}

\begin{verbatim}
## [1] "678-321-9876"
\end{verbatim}

\begin{Shaded}
\begin{Highlighting}[]
\FunctionTok{str\_extract}\NormalTok{(c, phone\_regex\_pattern)}
\end{Highlighting}
\end{Shaded}

\begin{verbatim}
## [1] "912-835-4756"
\end{verbatim}

7.) For lines below, extract the domains (ie the part of the address
after @)

\begin{Shaded}
\begin{Highlighting}[]
\NormalTok{d}\OtherTok{=}\StringTok{"jimmy.halibut@gmail.com"}
\NormalTok{e}\OtherTok{=}\StringTok{"His address is:  c.brown@hopeles.org, do write him"}
\NormalTok{f}\OtherTok{=}\StringTok{"h.potter@hogwarts.edu is bouncing back on me, I wonder why?"}

\NormalTok{email\_domain\_regex }\OtherTok{=} \StringTok{"@(}\SpecialCharTok{\textbackslash{}\textbackslash{}}\StringTok{w+}\SpecialCharTok{\textbackslash{}\textbackslash{}}\StringTok{.}\SpecialCharTok{\textbackslash{}\textbackslash{}}\StringTok{w+)}\SpecialCharTok{\textbackslash{}\textbackslash{}}\StringTok{W?"}
\FunctionTok{str\_extract}\NormalTok{(d, email\_domain\_regex, }\AttributeTok{group =} \DecValTok{1}\NormalTok{)}
\end{Highlighting}
\end{Shaded}

\begin{verbatim}
## [1] "gmail.com"
\end{verbatim}

\begin{Shaded}
\begin{Highlighting}[]
\FunctionTok{str\_extract}\NormalTok{(e, email\_domain\_regex, }\AttributeTok{group =} \DecValTok{1}\NormalTok{)}
\end{Highlighting}
\end{Shaded}

\begin{verbatim}
## [1] "hopeles.org"
\end{verbatim}

\begin{Shaded}
\begin{Highlighting}[]
\FunctionTok{str\_extract}\NormalTok{(f, email\_domain\_regex, }\AttributeTok{group =} \DecValTok{1}\NormalTok{)}
\end{Highlighting}
\end{Shaded}

\begin{verbatim}
## [1] "hogwarts.edu"
\end{verbatim}

\end{document}
